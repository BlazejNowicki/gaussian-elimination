\documentclass[12pt, letterpaper]{article}
\usepackage[margin=1in]{geometry}
\usepackage{graphicx}
\usepackage{float}
\usepackage[T1]{fontenc}
\usepackage[polish]{babel}
\usepackage[utf8]{inputenc}
\usepackage{amsmath}
\usepackage{listings, newtxtt}


\graphicspath{{images/}}
\lstset{basicstyle=\ttfamily, keywordstyle=\bfseries}

\title{Teoria Współbieżności - lab 5 \\ Eliminacja Gaussa}
\author{Błażej Nowicki}
\date{\today}
\begin{document}
\maketitle
\section{Wstęp teoretyczny}
\subsection{Niepodzielne zadania obliczeniowe}

Metoda eliminacji Gaussa jest to algorytm wykorzystywany do m. in.
rozwiązywania układów liniowych. Głównym krokiem jest sprowadzenie macierzy do postaci schodkowej
górnej. Można to osiągnąć odejmując wielokrotności kolejnych wierszy macierzy od wierszy poniżej
aż pod przekątną zostaną same zer.
Mamy daną przykładową macierz z wektorem wyrazów wolnych:

$$
	\begin{bmatrix}
		M_{1,1} & M_{1,2} & M_{1,3} \\
		M_{2,1} & M_{2,2} & M_{2,3} \\
		M_{3,1} & M_{3,2} & M_{3,3} \\
	\end{bmatrix}
	\begin{bmatrix}
		 M_{1,4} \\
		 M_{2,4} \\
		 M_{3,4} \\
	\end{bmatrix}
$$

Można wyróżnić trzy powtarzające się typy operacji:
\begin{itemize}
    \item $ A_{i, k} $ - wykonie dzielenia $ M_{k,i} / M_{i,i} $ aby uzyskać współczynnik $ m_{k,i} $ 
    \item $ B_{i,j,k} $ - mnożenie wyrazów $ M_{i,j} $ i $ m_{k,i} $, w celu uzyskania różnicy $ n_{k,j,i} $
    \item $ C_{i, j, k} $ - odejmownie $ n_{k,i} $ od $ M_{k,j} $
\end{itemize}

\subsection{Ciąg operacji}

Analizując jakie kroki należy wykonać można zauważyć powtarzający się schemat.
Dla każdego  wiersza najpierw wyznaczamy współczynnik - operacja $A$, a następnie 
na zmianę  $B$ i $C$ dla każdej kolumny. Efektywnie odejmujemy jeden wiersz od drugiego z wyznaczonym współczynnikiem. \\
Można tą operację oznaczyć jako $p_{i,k}$, gdzie: \\
$ p_{i,k} $ - odjęcie i-tego wiersza od k-tego ze współczynnikiem $M_{k,i}/M_{i,i}$ \\
$ p_{i,k} = ( A_{i,k}, B_{i,i,k}, C_{i,i,k}, B_{i,i+1,k}, C_{i,i+1,k},\ldots,B_{i, s+1, k}, C_{i,s+1,k}) $ \\
$s$ - rozmiar macierzy \\
Wtedy ciąg wszystkich operacji można zapisać jako: \\
$ (p_{1,2}, p_{1,3}, \ldots, p_{1,s},\\
p_{2,3}, p_{2,3}, \ldots, p_{2, s},\\
\ldots\\
p_{s-2, s-1}, p_{s-2, s} \\
p_{s-1, s}
) $

\subsection{Alfabet}

$ \Sigma = \{ A_{i,k} | i \subseteq \{1, \ldots, s-1\}, k \subseteq \{ i+1, \ldots, s \}\} \cup \\
\{ B_{i,j,k} | i \subseteq \{1, \ldots, s-1\}, k \subseteq \{ i+1, \ldots, s \}, j \subseteq \{ i, \ldots, s+1 \} \cup \\
\{ C{i,j,k} | i \subseteq \{1, \ldots, s-1\}, k \subseteq \{ i+1, \ldots, s \}, j \subseteq \{ i, \ldots, s+1 \}\}$

\subsection{Relacja zależności}

$D_1 = \{ (A_{i,k}, B_{i, j, k}) | i \subseteq \{ 1, \ldots, s-1 \},j \subseteq \{ i, \ldots, s+1 \}, k \subseteq \{i+1, \ldots, s-1 \}\}$ \\
$D_2 = \{ (B_{i,j,k}, C_{i,j,k}) | i \subseteq \{ 1, \ldots, s-1 \},j \subseteq \{ i, \ldots, s+1 \}, k \subseteq \{i+1, \ldots, s-1 \}\}$ \\
$D_3 = \{ (C_{i,j,k}, A_{i,j,k}) | \}$ \\


\subsection{Graf zależności Diekerta}
\subsection{Klasy Foaty}

\section{Implementacja}
\end{document}